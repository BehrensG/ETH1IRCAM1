\documentclass[10pt,a4paper]{article}
\usepackage[utf8]{inputenc}
\usepackage[english]{babel}
\usepackage[english]{isodate}
\usepackage[parfill]{parskip}
\usepackage{xcolor}
\usepackage{geometry}
\usepackage{hyperref}
\usepackage{lipsum}
\usepackage{setspace}

\definecolor{darkblue}{RGB}{0,0,100}

\geometry{
    left=2cm,
    right=2cm,
    top=2cm,
    bottom=2cm,
}

\hypersetup{
    colorlinks,
    linkcolor={red!50!black},
    citecolor={blue!50!black},
    urlcolor={blue!80!black}
}

\begin{document}

\part*{\textcolor{darkblue}{SCPI}}

\section*{\textcolor{darkblue}{\textbf{Introduction to SCPI}}}

SCPI, or Standard Commands for Programmable Instruments, is a standardized language used for controlling
and communicating with test and measurement instruments. It provides a unified framework for remote instrument
control, ensuring interoperability across different manufacturers and models. 

For more additional informations see:\\~\\
\href{https://en.wikipedia.org/wiki/Standard_Commands_for_Programmable_Instruments}{Standard Commands for Programmable Instruments}\\
\href{https://rfmw.em.keysight.com/spdhelpfiles/33500/webhelp/us/content/__I_SCPI/00%20scpi_introduction.htm}{Introduction to the SCPI Language}\\
\href{https://www.jaybee.cz/scpi-parser/}{SCPI parser library v2}\\

\subsection*{\textcolor{darkblue}{\textbf{SCPI Syntax}}}
The command syntax format is illustrated below:
\begin{itemize}
    \item \textbf{Command Header}: Every SCPI command begins with a command header, which typically consists of one or more letters followed by a colon (:).
     This header identifies the function or operation that the command instructs the instrument to perform. For example, \textbf{SYSTem:CONFigure:VOLTage}
      instructs the instrument to configure the voltage measurement settings.
      \item \textbf{Subsystems and Parameters}: SCPI commands often include hierarchical subsystems and parameters. These subsystems represent different functional 
      blocks or modules within the instrument, and parameters specify specific settings or values associated with those subsystems. For instance, in the command
      \textbf{SENS:VOLT:DC:RANG 10}, \textbf{SENS} represents the sensor subsystem, \textbf{VOLT} denotes voltage measurement, \textbf{DC} specifies the measurement
      type (direct current), and \textbf{RANG 10} sets the measurement range to 10 volts.
    \item \textbf{Query Commands}: SCPI supports query commands, which are used to retrieve instrument status or measurement data. Query commands typically end with
     a question mark (?). For example, \textbf{MEASure:VOLTage?} queries the instrument for the current voltage measurement.
\end{itemize}

\subsection*{\textcolor{darkblue}{SCPI Command Termination}}

\part*{\textcolor{darkblue}{Commands by Subsystem}}
\begin{spacing}{2}
    \hyperref[sec:SystemCommunicate]{\textbf{\large\textcolor{darkblue}{SYSTem:COMMunicate}}}\\
    \hyperref[sec:LeptonSystem]{\textbf{\large\textcolor{darkblue}{LEPton:SYSTem}}}
\end{spacing}
\subsubsection*{LEPton:OEM}
\subsubsection*{LEPton:VIDeo}

\section*{\label{sec:SystemCommunicate}System Communication Subsystem}
\section*{\label{sec:LeptonSystem}Lepton System Subsystem}

\end{document}