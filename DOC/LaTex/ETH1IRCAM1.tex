\documentclass[10pt,a4paper]{article}
\usepackage[utf8]{inputenc}
\usepackage[english]{babel}
\usepackage[english]{isodate}
\usepackage[parfill]{parskip}
\usepackage[table]{xcolor}
\usepackage{geometry}
\usepackage{hyperref}
\usepackage{lipsum}
\usepackage{setspace}
\usepackage{multirow}
\usepackage{tabularx}
\usepackage{enumitem}

\definecolor{darkblue}{RGB}{0,0,100}

\geometry{
    left=2cm,
    right=2cm,
    top=2cm,
    bottom=2cm,
}

\hypersetup{
    colorlinks,
    linkcolor={red!50!black},
    citecolor={blue!50!black},
    urlcolor={blue!80!black}
}

\begin{document}

\part*{\textcolor{darkblue}{SCPI}}

\section*{\textcolor{darkblue}{\textbf{Introduction to SCPI}}}

SCPI, or Standard Commands for Programmable Instruments, is a standardized language used for controlling
and communicating with test and measurement instruments. It provides a unified framework for remote instrument
control, ensuring interoperability across different manufacturers and models. 

For more additional informations see:\\~\\
\href{https://en.wikipedia.org/wiki/Standard_Commands_for_Programmable_Instruments}{Standard Commands for Programmable Instruments}\\
\href{https://rfmw.em.keysight.com/spdhelpfiles/33500/webhelp/us/content/__I_SCPI/00%20scpi_introduction.htm}{Introduction to the SCPI Language}\\
\href{https://www.jaybee.cz/scpi-parser/}{SCPI parser library v2}\\

\subsection*{\textcolor{darkblue}{\textbf{SCPI Syntax}}}
The command syntax format is illustrated below:
\begin{itemize}
    \item \textbf{Command Header}: Every SCPI command begins with a command header, which typically consists of one or more letters followed by a colon (:).
     This header identifies the function or operation that the command instructs the instrument to perform. For example, \textbf{SYSTem:CONFigure:VOLTage}
      instructs the instrument to configure the voltage measurement settings.
      \item \textbf{Subsystems and Parameters}: SCPI commands often include hierarchical subsystems and parameters. These subsystems represent different functional 
      blocks or modules within the instrument, and parameters specify specific settings or values associated with those subsystems. For instance, in the command
      \textbf{SENS:VOLT:DC:RANG 10}, \textbf{SENS} represents the sensor subsystem, \textbf{VOLT} denotes voltage measurement, \textbf{DC} specifies the measurement
      type (direct current), and \textbf{RANG 10} sets the measurement range to 10 volts.
    \item \textbf{Query Commands}: SCPI supports query commands, which are used to retrieve instrument status or measurement data. Query commands typically end with
     a question mark (?). For example, \textbf{MEASure:VOLTage?} queries the instrument for the current voltage measurement.
\end{itemize}

\subsection*{\textcolor{darkblue}{SCPI Command Termination}}

\part*{\textcolor{darkblue}{Commands by Subsystem}}
\begin{spacing}{2}
    \hyperref[sec:SystemCommunicate]{\textbf{\large\textcolor{darkblue}{SYSTem:COMMunicate}}}\\
    \hyperref[sec:LeptonSystem]{\textbf{\large\textcolor{darkblue}{LEPton:SYSTem}}}\\
    \hyperref[sec:LeptonOem]{\textbf{\large\textcolor{darkblue}{LEPton:OEM}}}\\
    \hyperref[sec:LeptonVideo]{\textbf{\large\textcolor{darkblue}{LEPton:VIDeo}}}\\
\end{spacing}

\section*{\label{sec:SystemCommunicate}System Communication Subsystem}

\section*{\label{sec:LeptonSystem}Lepton System Subsystem}

%-------------------------------------------------------------------------------------------------------------------

\subsection*{\textcolor{darkblue}{LEPton:SYSTem:STATus?}}

\vspace{12pt}

This command returns the system status.

    \begin{table}[h]
        \centering
        \begin{tabularx}{\textwidth}{|X|X|}
            \hline
            \rowcolor{gray!30} 
            Parameters & Result \\
            \hline
            NONE &  0, 1, 2, 3, 4\\
            \hline
            \multicolumn{2}{|X|}{
                \begin{itemize}[label={}, leftmargin=*]
                    \item 0 - Ready
                    \item 1 - Initializing
                    \item 2 - Low power model
                    \item 3 - going into stanby
                    \item 4 - flat field calibration (FFC) in process
                \end{itemize}
            } \\
            \hline
        \end{tabularx}
    \end{table}

%-------------------------------------------------------------------------------------------------------------------

\subsection*{\textcolor{darkblue}{LEPton:SYSTem:UPTIMe?}}

\vspace{12pt}

This command returns the current uptime in milliseconds. The uptime is the time since the
camera was brought out of Standby. The uptime counter is implemented as a 32-bit counter
and as such will rollover after the maximum count of 0xFFFFFFFF (1193 hours) is reached and restart at 0x00000000.

    \begin{table}[h]
        \centering
        \begin{tabularx}{\textwidth}{|X|X|}
            \hline
            \rowcolor{gray!30} 
            Parameters & Result \\
            \hline
            NONE &  Up time in miliseconds as uint32 value. The value is be displayed as decimal.\\
            \hline
        \end{tabularx}
    \end{table}

%-------------------------------------------------------------------------------------------------------------------

\subsection*{\textcolor{darkblue}{LEPton:SYSTem:AUX:TEMPerature?}}

\vspace{12pt}

This command returns the Lepton Camera’s AUX Temperature.

    \begin{table}[h]
        \centering
        \begin{tabularx}{\textwidth}{|X|X|}
            \hline
            \rowcolor{gray!30} 
            Parameters & Result \\
            \hline
            NONE &  AUX temperature as float value.\\
            \hline
        \end{tabularx}
    \end{table}

    \begin{itemize}
        \item The default temperature value is in kelvin. It is possible to select a different scale using \\ 
                \textcolor{darkblue}{SYSTem:TEMPerature:UNIT} command.
    \end{itemize}

%-------------------------------------------------------------------------------------------------------------------
\newpage

\subsection*{\textcolor{darkblue}{LEPton:SYSTem:FPA:TEMPerature?}}

\vspace{12pt}

This command returns the Lepton Camera’s FPA Temperature.

    \begin{table}[h]
        \centering
        \begin{tabularx}{\textwidth}{|X|X|}
            \hline
            \rowcolor{gray!30} 
            Parameters & Result \\
            \hline
            NONE &  FPA temperature as float value.\\
            \hline
        \end{tabularx}
    \end{table}

    \begin{itemize}
        \item The default temperature value is in kelvin. It is possible to select a different scale using \\ 
                \textcolor{darkblue}{SYSTem:TEMPerature:UNIT} command.
    \end{itemize}

%-------------------------------------------------------------------------------------------------------------------

\subsection*{\textcolor{darkblue}{LEPton:SYSTem:FRAMe:AVERage:RUN}}

\vspace{12pt}

This command executes the average frames command. Executing this command causes the camera to sum
together a number of frames, divide the summed frame by the number of frames summed and generate a result
frame containing the average of the summed frames. For Lepton 3.0 and 3.5, the number of frames is fixed at 8.

    \begin{table}[h]
        \centering
        \begin{tabularx}{\textwidth}{|X|X|}
            \hline
            \rowcolor{gray!30} 
            Parameters & Result \\
            \hline
            NONE &  NONE \\
            \hline
        \end{tabularx}
    \end{table}

%-------------------------------------------------------------------------------------------------------------------

\subsection*{\textcolor{darkblue}{LEPton:SYSTem:SN?}}

\vspace{12pt}

This command returns the Lepton Camera’s Customer serial number as a 32-byte character string. The Customer
Serial Number is a (32 byte string) identifier unique to a specific configuration of module.

\begin{table}[h]
    \centering
    \begin{tabularx}{\textwidth}{|X|X|}
        \hline
        \rowcolor{gray!30} 
        Parameters & Result \\
        \hline
        NONE &  String \\
        \hline
    \end{tabularx}
\end{table}

%-------------------------------------------------------------------------------------------------------------------

\subsection*{\textcolor{darkblue}{LEPton:SYSTem:SCENe:STATistics?}}

\vspace{12pt}

This command returns the current scene statistics for the video frame defined by the SYS ROI \\
(\textcolor{darkblue}{LEPton:SYSTem:SCENe:ROI}).
The statistics captured are scene mean intensity in counts, minimum and maximum intensity in counts, and the
number of pixels in the ROI. Lepton scene intensities range from 0 to 16383. The range drops to 0 to 255 when in
8-bit AGC mode. When TLinear mode is enabled (available in the Radiometric releases),
the camera output represents temperature values, and the scene statistics are reported in Kelvin x 100.

\newpage
\section*{\label{sec:LeptonOem}Lepton OEM Subsystem}
\section*{\label{sec:LeptonVideo}Lepton Video Subsystem}

\end{document}